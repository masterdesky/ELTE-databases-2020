%----------------------------------------------------------------------------------------
%	SLIDE 2.
%----------------------------------------------------------------------------------------
\begin{frame}
\frametitle{Amdahl's law}

\begin{itemize}
	\item<1-> Quantifies the theoretical magnitude of speedup due to parallelization.
	\item<2-> We can split a whole arbitrary algorithm (denoting with $1$) into to parts:
	\begin{equation} \label{eq:1}
		1 = S + P,
	\end{equation}
	where $S$ denotes the fraction of runtime of the sequentially and $P$ the parallel solved part.
	\item<3-> Amdahl's law now can be formulated as the following:
	\begin{equation} \label{eq:2}
		Q_{\text{speed up}} \left( N \right)
		=
		\frac{1}{S + \frac{P}{N}}
		=
		\frac{1}{\left( 1 - P \right) + \frac{P}{N}},
	\end{equation}
	Where $N$ is the number of parallel threads, and the runtime $S$ was expressed in the denominator using equation \eqref{eq:1}.
\end{itemize}

\end{frame}